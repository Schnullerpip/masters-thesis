\chapter{Motivation}\label{motivation}
We have now seen some of the fundamental differences between \textit{Object Oriented Programming} and \textit{Data oriented Design}. After recognizing the existence of a \textit{memory gap} we got to know some of the memory units our modern computer architectures are composed of. This helped us understanding why the caching technologies we make use of today tolerate but do not thrive on the real world metaphors we use to design our data models.\\
We learned that DoD offers methods that comply to our modern hardware, by trading abstraction as well as readability in exchange for performance. So in terms of utilizing the memory hierarchy it is inarguable superior to OOP. But we have also identified abstraction as one of the most important concepts in programming \mcp{bryant}{24} and to be one of the most important skills a programmer can have, since it is directly linked to a humans capability to solve a problem. We came to an understanding, that the abstraction model, that OOP inherits to us is widely accepted and applied in the industry, because it is intuitive and easy to learn.\\
Even when one is ready to abandon OOP it will persist. The industry is known for its reluctance to change. Implementing a new programming language into a developer team might be beneficial in long term, but comes with less to zero temporary productivity and initial training. Also DoD requires an understanding of hardware concerns, that novices and fresh graduates might not have. Most projects and companies are not even that dependent on high performance code, instead rely on solutions, that are quickly developed and easy to maintain and for the same reasons even the games industry won't solely rely on DoD probably ever. We also depend on gameplay programmers or level designers who will interact with an engine heavily but should not have to think about the underlying hardware all the time \mcp{fabian}{260}. The point is: No matter how much better other programming paradigms are on certain viewpoints and for specific tasks, OOP has its \textit{raison d'etre}. Instead of trying to sort it out we might just try to find a way to get the best out of both worlds.

\subsubsection{The best out of both worlds}
We already determined that DoD and OOP can get along to certain extends \refsecp{rtasl}. The more we want to rely on DoD to obtain performance boosts however, the more we will dismantle our abstraction step by step. Ideally we could keep whats best about OOP and still have optimal performance as though we had implemented our idea using DoD.\\
As mentioned before in \refsec{oop_bad_abstraction} DoD is not inferior to OOP in terms of maintainability per se. DoD's greatest disadvantage is, that it doesn't allow us to transfer a problem into code, the way we perceive it in the real world. The intuitive and thus advantageous abstraction model that comes with OOP is lost. On the other hand better performance makes a strong case for DoD, especially for game developers.\\
In conclusion what we want is the real world metaphors coming with OOP as well as the performance benefits coming from a data layout, that facilitates optimal cache utilization. Still, the question on how it is possible to achieve both things, remains.

\subsection{Native language support for DoD principles / ISPC / JAI}\label{nat_lan_sup}
There are languages in existence and under development, that aim to provide native support for SOA/AOSOA data structures!

\subsubsection{Intel's ISPC}
\begin{wrapfigure}[11]{r}{0.34\textwidth}
\vspace{-0.8cm}
\begin{lstlisting}[language=C++,name={ISPC's native SOA support},morekeywords={soa}, label={ispc_npc}]
struct npc_group{
	float x, y, z;
	float v_x, v_y, v_z;
};

soa<128> npc_group pos_and_vel;
soa<16> npc_group aosoa_pos_and_vel[8];
\end{lstlisting}
\end{wrapfigure}
The \textit{Intel SPMD Program Compiler} (ISPC) is specifically designed to support quick and easy development of \textit{Single Program Multiple Data} \textit{SPMD} applications, making use of implicit \textit{Single Instruction Multiple Data} (SIMD) vector units \mc{ispc}.\\
Those instructions depend on SOA data layouts, thus the language provides a \textit{soa} keyword to automatically transform an AOS defined struct into a SOA format. In \refcode{ispc_npc} an AOS \textit{npc\_group} is defined in line 1. In line 6 the \textit{pos\_and\_vel} is defined as a SOA holding 128 consecutive \textit{x}, \textit{y}, \textit{z}, \textit{v\_x}, \textit{v\_y}, and \textit{v\_z} respectively. This also easily enables for AOSOA format as can be seen in line 7.

\subsubsection{Jonathan Blow and JAI}
\begin{wrapfigure}[6]{r}{0.34\textwidth}
\vspace{-0.8cm}
\begin{lstlisting}[language=C++,name={JAI's native SOA support},morekeywords={SOA}, label={jai_npc}]
npc_group :: struct SOA{
	xyz : [3] float;
	vel : [3] float;
};
\end{lstlisting}
\end{wrapfigure}
A prominent game developer and critic of the C++ language Jonathan Blow for example is working on the \textit{JAI} programming language. There is currently no official documentation to it and it is unknown when the language will be released to public, but some of its features and its design goals are already well known. One of the highly anticipated features is automatic AOS to SOA conversion done by the compiler using nothing but a single keyword. There is no official documentation and information presented here originates only from the various online video talks Blow provides occasionally. In \mc{jonblow} the \textit{SOA} keyword is introduced as a typespecifier when creating a struct, automatically informing the compiler to store the struct's members in a SOA fashion and granting correct access to them \refcodep{jai_npc}.

\subsection{High level abstraction hiding DoD}
However we already know, that introducing new languages/technologies into a functioning industry is mostly viewed as a cost factor and since C++ is the most prominent language in the game development industry, we can not expect to see a lot of native language support for DoD principles like SOA in the near future (unless the ISO C++ committee decides in its favor).\\
One possible solution could be to provide high level abstraction containers, that internally work with data oriented concepts. In his online blog article \textit{Implementing a semi-automatic structure-of-arrays data container} \mc{reinalter} Stefan Reinalter introduces a possible implementation for such mechanisms. Template meta programming is a way of interacting with the compiler and can to a certain extend overcome the inherent conflict between OOP and DoD, but Reinalter states that:
\begin{quote}
	\textit{I really would like to have a fully automatic implementation, but I don’t believe that’s possible without compiler support.} \mc{reinalter}
\end{quote}
Even when we can provide high level data containers, that implement a cache friendly data layout, it can not completely decouple the process of modeling the reality into code from reflecting about its data layout considering optimal hardware utilization. The high level containers in the end still need to be used correctly and based on implementation may require to define the relevant records dependent on it (for example with macros), because due to missing \textit{reflection} features, we can't iterate a record's members.\\
If possible an ideal solution would be uncorrupted high abstraction code, that somehow translates to high performance code. The relevant keyword here is \textit{translate}. Compilers normally do these kinds of tasks. The question arises whether we could utilize compiler technology to accomplish our goal.