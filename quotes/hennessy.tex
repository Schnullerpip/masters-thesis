\section{hennessy}
\subsection{principle of locality}
\textit{The	most important program property that we regularly exploit is the principle of
	locality: Programs tend to reuse data and instructions they have used recently. A
	widely held rule of thumb is that a program spends 90\% of its execution time in
	only 10\% of the code. An implication of locality is that we can predict with reasonable
	accuracy what instructions and data a program will use in the near future
	based on its accesses in the recent past. The principle of locality also applies to
	data accesses, though not as strongly as to code accesses.
	Two different types of locality have been observed. Temporal locality states
	that recently accessed items are likely to be accessed in the near future. Spatial
	locality says that items whose addresses are near one another tend to be referenced
	close together in time.}
\mcp{hennessy}{38}

