\outline{This work will prospect the possibility of using compiler technology as a mediator between the conflicting programming paradigms \textit{OOP} and \textit{SOA}.\\While Object oriented programming is often praised for its benefits on abstraction and maintainability, it encourages programmers to design inefficient datalayouts. Specifically in game engineering, where performance is a constitutive factor for a product's success, data oriented solutions are on the rise. While it is debated, wether or not performant data layouts inevitably entail challenging maintenance, surely the base concepts of \textit{objects} are well observable in a game. Therefore this will be an attempt to make object oriented programming a valid option for ever rising demands on performance.\\
A prototypical implementation of a source-to-source transformation tool called \textit{COOP} (\textbf{C}ache friendly \textbf{O}bject \textbf{O}riented \textbf{P}rogramming), developed in the Clang tooling environment, will determine wether or not compiler technology can be used to achieve a performance optimization on a classically OOP abidant target source code base.}